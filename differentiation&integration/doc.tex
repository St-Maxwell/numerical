\documentclass{ctexart}
\usepackage{amsmath}
\usepackage{graphicx,booktabs,array,siunitx}

\title{数值微分和积分}
\author{St Maxwell}
\begin{document}
\maketitle

\section{数值微分}
数值微分似乎没什么花样,好像都是有限差分的方法。

由函数的Taylor级数(假定函数二阶连续可微)
\[
f(x+h) = f(x) + hf'(x) + \frac{h^2}{2}f''(c)
\]
其中$c\in(x,x+h)$。该式变形得到\textbf{二点前向差分公式}
\[
f'(x) = \frac{f(x+h) - f(x)}{h} - \frac{h}{2}f''(c)
\]

在计算中,当$h$很小时,上式前一项近似为导数
\[
f'(x) \approx \frac{f(x+h) - f(x)}{h}
\]
后一项可看作误差。显然误差与步长$h$近似成正比,因此可以减小$h$从而减小误差。二点前向差分公式是近似一阶导数的一阶方法。如果误差是$O(h^n)$,我们把该公式称为$n$阶近似。

然后导出二阶近似公式。由函数的函数的Taylor级数(假定函数三阶连续可微)
\[
f(x+h) = f(x) + hf'(x) + \frac{h^2}{2}f''(x) + \frac{h^3}{6}f'''(c_1)
\]
\[
f(x-h) = f(x) - hf'(x) + \frac{h^2}{2}f''(x) - \frac{h^3}{6}f'''(c_2)
\]
其中$x-h<c_2<x<c_1<x+h$。两式相减得到
\[
f'(x) = \frac{f(x+h) - f(x-h)}{2h} - \frac{h^2}{12}f'''(c_1) - \frac{h^2}{12}f'''(c_2)
\]
由推广中值定理,可合并后两项,得到二阶公式(\textbf{三点中心差分公式})
\[
f'(x) = \frac{f(x+h) - f(x-h)}{2h} - \frac{h^2}{6}f'''(c)
\]
其中$x-h<c<x+h$。

我们以求$f(x) = \mathrm{e}^x$在$x=1$处的导数为例,结果见表。比较两个表中的结果,显然二阶公式的误差具有更快的下降速度,且最小误差也小于一阶公式。表明二阶公式更具有优越性。此外,由于所有导数计算公式不可避免存在相近数字相减的情况。因此不断减小$h$,会导致数值精度的损失,误差反而增加。对三点中心差分公式,当
\[
h = (3\epsilon_{mach}/M)^{1/3}
\]
时误差能达到极小。在双精度中约为$\epsilon_{mach}^{1/3} \approx 10^{-5}$。

\begin{table}[h]
\centering
$\begin{array}{ccccc}
\hline
h & \text{一阶} & \text{error} & \text{二阶} & \text{error} \\
\hline
10^{-1}  & 2.8588419548739 &  0.1405601264148 & 2.7228145639474 &  0.0045327354884 \\
10^{-2}  & 2.7319186557871 &  0.0136368273281 & 2.7183271333827 &  0.0000453049237 \\
10^{-3}  & 2.7196414225332 &  0.0013595940742 & 2.7182822815057 &  0.0000004530467 \\
10^{-4}  & 2.7184177470829 &  0.0001359186239 & 2.7182818329896 &  0.0000000045306 \\
10^{-5}  & 2.7182954199567 &  0.0000135914977 & 2.7182818285176 &  0.0000000000586 \\
10^{-6}  & 2.7182831874306 &  0.0000013589716 & 2.7182818282956 & -0.0000000001635 \\
10^{-7}  & 2.7182819684057 &  0.0000001399467 & 2.7182818285176 &  0.0000000000586 \\
10^{-8}  & 2.7182818218563 & -0.0000000066028 & 2.7182818218563 & -0.0000000066028 \\
10^{-9}  & 2.7182820439009 &  0.0000002154419 & 2.7182818218563 & -0.0000000066028 \\
10^{-10} & 2.7182833761685 &  0.0000015477095 & 2.7182811557225 & -0.0000006727366 \\
\hline
\end{array}$
\end{table}

\section{数值积分}
\subsection{牛顿-科特斯公式}
几种数值积分方法的思路都是先对考察区间上的点进行多项式插值,然后对多项式进行积分。

\subsubsection{梯形法则}
对$f(x)$在区间两端点拟合,使用拉格朗日公式,得到具有误差项的插值多项式
\[
f(x) = y_0\frac{x - x_1}{x_0 - x_1} + y_1\frac{x - x_0}{x_1 - x_0} + \frac{(x - x_0)(x - x_1)}{2!} f''(c_x) = P(x) + E(x)
\]

在区间内对函数积分
\[
\int_{x_0}^{x_1} f(x)\,\mathrm{d}x = \int_{x_0}^{x_1} P(x)\,\mathrm{d}x + \int_{x_0}^{x_1} E(x)\,\mathrm{d}x
\]
第一个积分结果为
\[
\int_{x_0}^{x_1} P(x)\,\mathrm{d}x = y_0\frac{h}{2} + y_1\frac{h}{2} = h\frac{y_0 + y_1}{2}
\]
误差项为
\[
\int_{x_0}^{x_1} E(x)\,\mathrm{d}x = -\frac{h^3}{12}f''(c)
\]

因此\textbf{梯形法则}
\[
\int_{x_0}^{x_1} f(x)\,\mathrm{d}x = h\frac{y_0 + y_1}{2} - \frac{h^3}{12}f''(c)
\]
其中$h=x_1-x_0$,$x_0<c<x_1$。

\subsubsection{辛普森法则}
对三点进行插值,将之前的一阶公式换成二阶
\[
\begin{split}
f(x) &= y_0\frac{(x - x_1)(x - x_2)}{(x_0 - x_1)(x_0 - x_2)} + y_1\frac{(x - x_0)(x - x_2)}{(x_1 - x_0)(x_1 - x_2)} \\
     &+ y_2\frac{(x - x_0)(x - x_1)}{(x_2 - x_0)(x_2 - x_1)} + \frac{(x - x_0)(x - x_1)(x - x_2)}{3!}f'''(c_x) \\
     &= P(x) + E(x)
\end{split}
\]

两端积分,得到\textbf{辛普森法则}
\[
\int_{x_0}^{x_2} f(x)\,\mathrm{d}x = \frac{h}{3}(y_0 + 4y_1 + y_2) - \frac{h^5}{90}f^{(4)}(c)
\]
其中$h=x_1-x_0=x_2-x_1$,$x_0<c<x_2$。

\subsubsection{复合牛顿-科特斯公式}
梯形和辛普森法则都是在非常小的区间上的积分。对于大的区间,可以将其分为很多个小区间积分并求和。为了计算
\[
\int_a^b f(x)\,\mathrm{d}x
\]
将区间$[a,b]$等分为多个小格点
\[
a = x_0 < x_1 < x_2 < \ldots < x_{m-1} < x_m = b
\]
对于所有$i$,有$h=x_{i+1}-x_i$。将所有小区间的积分以及误差项进行求和,得到\textbf{复合梯形公式}
\[
\int_a^b f(x)\,\mathrm{d}x = \frac{h}{2}\left(y_0 + y_m + 2\sum\limits_{i=1}^{m-1} y_i\right) - \frac{(b - a)h^2}{12}f''(c)
\]
其中$h=(b-a)/m$,$a<c<b$。

复合辛普森公式与之类似,其区间划分方式为
\[
a = x_0 < x_1 < \ldots < x_{2m-1} < x_{2m} = b
\]
对于所有$i$,有$h=x_{i+1}-x_i$。在每个长度为$2h$的区间$[x_{2i},x_{2i+2}](i=0,1,\ldots,m-1)$上分别使用辛普森法则,然后求和。\textbf{复合辛普森公式}有
\[
\int_a^b f(x)\,\mathrm{d}x = \frac{h}{3}\left(y_0 + y_{2m} + 4\sum\limits_{i=1}^m y_{2i-1} + 2\sum\limits_{i=1}^{m-1} y_{2i}\right) - \frac{(b - a)h^4}{180}f^{(4)}(c)
\]
其中$a<c<b$。

对比复合梯形公式和复合辛普森公式的误差项,很明显后者将具有更高的精度。以$f(x)=\ln{x}$为例,在区间$[1,2]$上计算积分(精确值为$2\ln{2}-1$)。将区间分为十段分别计算积分,结果如下。

\begin{table}[h]
\centering
$\begin{array}{ccc}
\hline
 & \text{积分} & \mathrm{error} \\
\hline
\text{复合梯形公式} & 0.3858779367458 & -0.000416424374 \\
\text{复合辛普森公式} & 0.3862943005944 & -0.000000060526 \\
\hline
\end{array}$
\end{table}

\subsection{龙贝格积分}
\textbf{\textit{Euler-MacLaurin}定理}:若积分公式$I^{(m)}$是$2m$阶公式$I(f)=I^{(m)}(h)+O(h^{2m})$,则公式
\[
I^{(m+1)}\Big(\frac{h}{2}\Big) = I^{(m)}\Big(\frac{h}{2}\Big) + \frac{I^{(m)}\left(\frac{h}{2}\right) - I^{(m)}(h)}{2^{2m} - 1}
\]
为$2m+2$阶公式,即有$I(f)=I^{(m+1)}+O(h^{2m+2})$。

龙贝格积分就是不断地组合低阶公式为高阶公式,从而计算积分的近似值。定义如下的步长序列:
\[
\begin{matrix}
h_1 = b - a \\
h_2 = \frac{1}{2}(b - a) \\
\vdots \\
h_j = \frac{1}{2^{j-1}}(b - a)
\end{matrix}
\]
定义近似公式$R_{j1}$使用步长$h_j$的复合梯形法则,因此$R_{j+i,1}$对应外推使$R_{j1}$步长减半的结果。对于$j=2,3,\ldots$,有
\[
R_{j1} = \frac{1}{2}R_{j-1,1} + h_j \sum\limits_{i=1}^{2^{j-2}} f(a + (2i - 1)h_j)
\]

第$jk$项由如下公式给出:
\[
R_{jk} = \frac{4^{k-1}R_{j,k-1} - R_{j-1,k-1}}{4^{k-1} - 1}
\]

由这两个递推公式,首先计算$R_{11}$,然后计算第二行$R_{12}$和$R_{22}$,再计算第三行,依此类推。收敛条件是该行与上一行的相邻对角线上的元素之差小于容差。选择与之前相同的例子,在$[1,2]$上计算积分$f(x)=\ln{x}$的近似值。结果如下(容差为$10^{-7}$):
\[
\begin{matrix}
  0.346573590279973 &                   &                   &                   &                   \\
  0.376019349194069 & 0.385834602165434 &                   &                   &                   \\
  0.383699509409442 & 0.386259562814567 & 0.386287893524509 &                   &                   \\
  0.385643909952095 & 0.386292043466313 & 0.386294208843096 & 0.386294309086248 &                   \\
  0.386131637744868 & 0.386294213675793 & 0.386294358356425 & 0.386294360729652 & 0.386294360932175
\end{matrix}
\]
因此积分的近似结果为$0.3862943609322$,误差为$-0.000000000188$。
















\end{document}
