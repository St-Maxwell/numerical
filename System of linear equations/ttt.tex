\documentclass{ctexart}
\usepackage{amsmath,mathtools}
\usepackage{graphicx,booktabs,array,listings}
\usepackage[T1]{fontenc}
\usepackage{xcolor}
\usepackage{lmodern}
\definecolor{mygreen}{rgb}{0,0.6,0}
\definecolor{mygray}{rgb}{0.5,0.5,0.5}
\definecolor{mymauve}{rgb}{0.58,0,0.82}

\title{方程组求解}
\author{St Maxwell}
\begin{document}
\maketitle
\lstset{
basicstyle=\footnotesize,        % the size of the fonts that are used for the code
  breakatwhitespace=false,         % sets if automatic breaks should only happen at whitespace
  breaklines=false,                 % sets automatic line breaking
  captionpos=b,                    % sets the caption-position to bottom
  commentstyle=\color{mygreen},    % comment style
  extendedchars=true,              % lets you use non-ASCII characters; for 8-bits encodings only, does not work with UTF-8
  keepspaces=true,                 % keeps spaces in text, useful for keeping indentation of code (possibly needs columns=flexible)
  keywordstyle=\color{blue},       % keyword style
  language=[95]Fortran,            % the language of the code
  numberstyle=\tiny\color{mygray}, % the style that is used for the line-numbers
  rulecolor=\color{black},         % if not set, the frame-color may be changed on line-breaks within not-black text (e.g. comments (green here))
  showspaces=false,                % show spaces everywhere adding particular underscores; it overrides 'showstringspaces'
  showstringspaces=false,          % underline spaces within strings only
  showtabs=false,                  % show tabs within strings adding particular underscores
  stepnumber=1,                    % the step between two line-numbers. If it's 1, each line will be numbered
  stringstyle=\color{mymauve},     % string literal style
  tabsize=4,                       % sets default tabsize to 2 spaces
  title=\lstname                   % show the filename of files
}

\section{Gauss消元法}
\[
\begin{cases}
2x-2y-z &= -2 \\
4x+y-2z &= 1 \\
-2x+y-z &= -3
\end{cases}
\]

对于以上的线性方程组,可以将其写成(增广)矩阵的形式:
\[
\left(
\begin{array}{c|c}
\begin{matrix}
2 & -2 & -1 \\
4 & 1 & -2 \\
-2 & 1 & -1
\end{matrix}&
\begin{matrix}
-2 \\
1 \\
-3
\end{matrix}
\end{array}
\right)
\]

使用Gauss消元法,首先通过矩阵初等变换进行消去,将左边变成上三角矩阵。
\[
\left(
\begin{array}{c|c}
\begin{matrix}
2 & -2 & -1 \\
0 &  5 &  0 \\
0 &  0 & -2 \\
\end{matrix}&
\begin{matrix}
-2 \\
 5 \\
-4
\end{matrix}
\end{array}
\right)
\]

之后进行回代,并将主元约化为$1$。
\[
\left(
\begin{array}{c|c}
\begin{matrix}
1 & 0 & 0 \\
0 & 1 & 0 \\
0 & 0 & 1 \\
\end{matrix}&
\begin{matrix}
1 \\
1 \\
2
\end{matrix}
\end{array}
\right)
\]

由此得到方程组的解。
\[
\begin{cases}
x=1 \\
y=1 \\
z=2
\end{cases}
\]

\section{fortran代码细节与LAPACK包}
代码中的增广矩阵是使用\lstinline!reshape!函数生成的。
\begin{lstlisting}[frame=trBL]
 real(kind=8) :: mat(4,3) = reshape( (/ 2,-2, -1,-2, &
                                        4, 1, -2, 1, &
                                       -2, 1, -1,-3 /), (/ 4,3 /) )
\end{lstlisting}
所以矩阵元(数组元素)的下标和通常是相反的。例如,上面的增广矩阵中$a_{21}=4$;而在数组中,它对应于\lstinline!mat(1,2)!。

要说明这一点的主要原因是,这个数组的定义方式与LAPACK包的\lstinline!dgesv!子程序对输入数组的要求是相反的。

\begin{lstlisting}[frame=trBL]
 PROGRAM Main
  IMPLICIT NONE
  REAL    :: A(3,3), b(3)
  INTEGER ::  i, j, pivot(3), ok

  ** Initialize A(:,:)
  ** Initialize b(:)

  CALL SGESV(3, 1, A, 3, pivot, b, 3, ok)    

  PRINT *, b    !! Print solution
 END PROGRAM
\end{lstlisting}

以上代码中的\lstinline!A(i,j)!即对应矩阵元$A_{ij}$。此时若依然使用\lstinline!reshape!函数,其矩阵元应按列依次定义。

正式的项目中应该还是考虑使用LAPACK包,所以以后定义矩阵就选择第二种方式,而且也符合数学上的习惯。























\end{document}